\documentclass[12pt]{article}

% PAQUETES IMPORTADOS
\usepackage[utf8]{inputenc}                     %input encoding
\usepackage[a4paper, margin=2.5cm]{geometry}    %margins
\usepackage{sectsty}                            %tamanos de cabeceras
\usepackage[spanish,es-tabla]{babel}            %espanol
\usepackage{fontspec}                           %fuentes
\usepackage{anyfontsize}                        %tamanos de fuentes
\usepackage[style=ieee]{biblatex}               %bibliografia
\usepackage{csquotes}                           %citas a bibliografia
\usepackage{nameref}                            %referencia a capitulos
\usepackage{amsmath}                            %inline maths
\usepackage{amssymb}                            %math symbols
\usepackage{graphicx}                           %imagenes
\usepackage{color}                              %colores de fuentes
\usepackage{hyperref}                           %enlaces
\usepackage{ulem}                               %subrayados decentes

% PROPIEDADES DE PAQUETES
\hypersetup{    % Color de enlaces (usados para referencias a capítulos)
    colorlinks,
    citecolor=black,
    filecolor=black,
    linkcolor=black,
    urlcolor=black
}

% ARCHIVOS Y CARPETAS AUXILIARES
\graphicspath{ {./images/} }
\addbibresource{biblio.bib}

% FUENTES
\renewcommand{\contentsname}{Índice de Contenidos}
\setmainfont{Times New Roman}
\sectionfont{\fontsize{20pt}{15}\selectfont}
\subsectionfont{\fontsize{18pt}{15}\selectfont}
\subsubsectionfont{\fontsize{16pt}{15}\selectfont}

\begin{document}

%---------------------------------------------------
% PORTADA
%---------------------------------------------------
\begin{titlepage}
\begin{flushright}
\hfill\parbox[r][][c]{0.5\textwidth}{
{\fontsize{14}{14}\selectfont Universidad de Valladolid \\
E.T.S Ingeniería Informática \\
Grado en Ingeniería Informática\\
Rama de Ingeniería de Software\\
Cuarto Curso}}

\end{flushright}

\vfill
\centering
{\fontsize{16pt}{1cm}\selectfont \textbf{Resumen del Libro de la Asignatura <<Planificación y Gestión de Proyectos>>: Software Project Management}}


\vfill
\begin{flushright}
\hfill\parbox[r][][c]{0.4\textwidth}{
{\fontsize{14pt}{14cm}\selectfont
Bayón Sanz, Miguel
}
}
\end{flushright}

\end{titlepage}

%---------------------------------------------------
% INDICE
%---------------------------------------------------
\tableofcontents
\newpage

%---------------------------------------------------
% CUERPO
%---------------------------------------------------

%---------------------------------------------------
% SECCION 1
%---------------------------------------------------
\newpage
\section{Introducción a la gestión de proyectos\\ Software}
\label{1.0.0}
\subsection{Introducción}
\label{1.1.0}

{Los proyectos de Software, al igual que otros proyectos, se basan en cumplir una serie de objetivos para satisfacer necesidades reales. Esto se hará encontrando a los inversores y sus objetivos. La gestión de proyectos tiene como finalidad cumplir sus objetivos, conociendo el estado del mismo proyecto en todo momento.}

\subsection{¿Por qué es importante la gestión de proyectos de Software?}
\label{1.2.0}

{Los proyectos de Software son el tipo de proyecto que más dinero mueve, implicando que una mala gestión haga que se pueda perder todo ese dinero. Según un estudio, de entre 13.522 proyectos:}

\begin{itemize}
    \item {El 66\% fracasa.}
    \item {El 82\% terminan tarde.}
    \item {El 43\% excede el presupuesto.}
\end{itemize}

\subsection{¿Qué es un proyecto?}
\label{1.3.0}

{Un proyecto es una \textbf{actividad planeada} que cuenta con la definición de todos los pasos y de su duración. Este cuenta también con una definición oficial establecida en el BSO ISO 10006 (1997): \textit{Proceso único consistente en un conjunto de actividades coordinadas y controladas con fechas de inicio y fin, promovido para conseguir un objetivo conforme a requisitos específicos incluyendo restricciones de tiempo, coste y recursos}.} \\

{El proyecto se encuentra en un término medio entre un trabajo rutinario y una investigación. Para diferenciarlos, debe cumplir los siguientes puntos:}

\begin{itemize}
    \item {Debe contener \textbf{tareas no rutinarias}.}
    \item {Requiere \textbf{planificación}.}
    \item {Debe cumplir \textbf{objetivos específicos}.}
    \item {Se debe completar en un \textbf{periodo de tiempo}.}
    \item {El trabajo se realiza \textbf{para otra persona}.}
    \item {Implica \textbf{varias especialidades}.}
    \item {Los empleados formarán un \textbf{grupo de trabajo temporal}.}
    \item {El trabajo se llevará a cabo en \textbf{varias fases}.}
    \item {Los \textbf{recursos} del proyecto estarán \textbf{limitados}.}
    \item {Debe ser un proyecto \textbf{grande y/o complejo}.}
\end{itemize}

{El tamaño del equipo es especialmente importante, porque implicará una mayor organización entre los participantes. Además, toda la experiencia ganada como equipo \textbf{se perderá al disolverse} el mismo (al ser grupos temporales).}

\subsection{Proyectos de Software vs otros tipos de proyectos}
\label{1.4.0}

{Algunas características de los proyectos de Software los hacen especialmente complicados:}

\begin{itemize}
    \item {\textbf{Invisibilidad}: el progreso no es inmediatamente visible.}
    \item {\textbf{Complejidad}: el proyecto de Software tiene gran complejidad por unidad monetaria.}
    \item {\textbf{Conformidad}: los desarrolladores se tienen que conformar con los requisitos de los clientes, que pueden ser cambiantes e inconsistentes.}
    \item {\textbf{Flexibilidad}: los sistemas Software están sujetos al cambio para acomodarse a elementos externos.}
\end{itemize}

\subsection{Gestión de contratos y gestión de proyectos técnicos}
\label{1.5.0}

{En los proyectos internos se desarrolla para la misma organización. En los externos, al haber un contrato entre cliente y proveedor, existirán dos gestores de proyectos: uno del lado del cliente y otro del lado del proveedor. El del cliente (gestor del contrato) se ocupará de comprobar que el proyecto cumpla con los plazos y se ajuste al presupuesto, mientras que el del proveedor se ocupará de las decisiones de carácter técnico del proyecto.}

\subsection{Actividades cubiertas por la gestión de proyectos de Software}
\label{1.6.0}

{Un proyecto de Software suele pasar por \textbf{tres procesos} para generar un nuevo sistema:}

\begin{itemize}
    \item {\textbf{Estudio de viabilidad}: se evalúa lo que se quiere conseguir con sus requisitos y se comprueba que el proyecto sea viable. Si el proyecto es grande, el estudio de viabilidad puede ser un proyecto en sí mismo.}
    \item {\textbf{Planificación}: si el proceso anterior tiene un resultado positivo puede comenzar la planificación. Para proyectos muy grandes, la planificación no se detalla desde el principio, sino que se genera un guion general para todo el proyecto y se detalla solamente la primera etapa, dejando el resto para cuando se llegue a ellas.}
    \item {\textbf{Ejecución de proyecto}: El proyecto se ejecuta, a veces con sub-secciones de diseño e implementación. El libro, además, recalca la diferencia entre planificación (detallado de las actividades a llevar a cabo para generar un producto) y el diseño (decisiones sobre la forma del producto final).}
\end{itemize}
\newpage

{Durante el proceso de implementación, surgirán actividades como las siguientes:}

\begin{itemize}
    \item {\textbf{Análisis de requisitos}: funciones o calidad mínima que serán requeridas por los usuarios del sistema.}
    \item {\textbf{Diseño de arquitectura}: Selección de los componentes (Software, Hardware o proceso de trabajo) que cumplirán cada requisito.}
    \item {\textbf{Diseño detallado}: Se diseñan independientemente las unidades que compondrán cada componente Software.}
    \item {\textbf{Programar y testear}: Escritura y depuración de cada unidad.}
    \item {\textbf{Integración}: Se prueban los componentes juntos.}
    \item {\textbf{Prueba de cualificación}: Se comprueba que todos los requisitos se cumplen.}
    \item {\textbf{Instalación}: El sistema trabaja en condiciones reales y/o para clientes.}
    \item {\textbf{Soporte de aceptación}: Resolución de errores en el sistema funcional. Una resolución de error puede ser un proyecto completo.}
\end{itemize}

\subsection{Planes, métodos y metodologías}
\label{1.7.0}

\begin{itemize}
    \item {\textbf{Metodología}: Conjunto de métodos.}
    \item {\textbf{Método}: Forma de trabajar o de organizar el trabajo.}
    \item {\textbf{Plan}: Organización de las tareas de un proyecto siguiendo una serie de métodos.}
\end{itemize}

\subsection{Algunas maneras de clasificar los proyectos de Software}
\label{1.8.0}

{Algunos factores a tener en cuenta al diseñar un proyecto son:}

\begin{itemize}
    \item {\textbf{Usuarios obligados contra usuarios voluntarios}: Un producto de uso empresarial contra un videojuego.}
    \item {\textbf{Sistemas de información contra sistemas embebidos}: Control de información contra control de máquinas.}
    \item {\textbf{Objetivos contra productos}: Solucionar un problema contra crear un nuevo sistema.}
\end{itemize}

\subsection{Stakeholders}
\label{1.9.0}

{Son aquellos que tienen un interés o una implicación directa en el proyecto o que se beneficiarán de él. Pueden ser:}

\begin{itemize}
    \item {Miembros del equipo del proyecto}
    \item {Externos al equipo del proyecto pero pertenecientes a la misma organización}
    \item {Externos a la empresa}
\end{itemize}

{Un buen líder de proyecto tiene que tratar de buscar los intereses de estos stakeholders y plasmarlos en el proyecto, buscando que todos obtengan beneficio de ello (La <<Theory W>> de Boehm and Ross, la situación \textit{win-win}).}

\subsection{Estableciendo objetivos}
\label{1.10.0}

{Algunos stakeholders serán los financiadores del proyecto y que además serán dueños del resultado y establecerán objetivos.}\\

{Los objetivos se definen como post-condiciones del proyecto que deben ser cumplidas para que el proyecto pueda tener éxito (por ejemplo, <<nuestro proyecto tendrá éxito si el cliente puede comprar productos online>>.}\\

{En caso de que varios stakeholders puedan tener derecho a la propiedad del proyecto, se establece una autoridad (generalmente un \textit{comité de dirección de proyecto}) con la potestad de establecer, vigilar y modificar objetivos. El líder de proyecto tendrá que reportarle los avances.}

\subsubsection{Sub-objetivos o metas}
\label{1.10.1}

{Al ser algunos objetivos muy generales (<<\textit{el proyecto será un éxito si reduce el consumo de energía}>>), se suelen dividir en sub-objetivos o metas fáciles de interpretar (<<\textit{para alcanzar este objetivo, será necesario...}>>).}\\

{Se suele definir un objetivo bien definido con el acrónimo \textbf{SMART}:}
\begin{itemize}
    \item {\textbf{S}pecific o específico: es concreto y está bien definido.}
    \item {\textbf{M}easurable o medible: debe ofrecer medidas (como \textit{reducir} o \textit{reducir en X unidad}) en lugar de conceptos abstractos (como \textit{mejorar}).}
    \item {\textbf{A}chievable o lograble.}
    \item {\textbf{R}elevant o relevante en cuanto al propósito del proyecto.}
    \item {\textbf{T}ime constrained o restringido en el tiempo: se tiene que poder completar en un tiempo determinado.}
\end{itemize}

\subsubsection{Medidas de efectividad}
\label{1.10.2}

{Las medidas de efectividad ofrecen una forma práctica de comprobar el cumplimiento de un objetivo.}

\subsection{Business case o Caso de negocio}
\label{1.11.0}

{El Business case forma parte del estudio de viabilidad y consiste en un análisis coste-beneficio y cuándo se conseguirá ese beneficio.}\\

{Los planes de proyecto deben asegurarse de que el Business case se mantiene intacto.}

%---------------------------------------------------
% BIBLIOGRAFIA
%---------------------------------------------------

\newpage
\section{Única bibliografía}
\nocite{*}
\begingroup
\renewcommand{\section}[2]{}%
\printbibliography
\endgroup


\end{document}